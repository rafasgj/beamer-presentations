\documentclass[aspectratio=169,14pt,usenames,dvipsnames]{beamer}

\usepackage[utf8]{inputenc}
\usepackage{enumitem}
\usepackage{calc}

\newcommand{\loadtheme}[1]{%
    \input{../themes/#1}%
}
\newcommand{\presentationlanguage}[1]{%
    \usepackage[#1]{babel}%
}

\newcommand{\usecodingsamples}{%
    \usepackage{listings}%
    %
% You must use package 'listings' for this environment.
%

% Python style for highlighting
\newcommand\pythonstyle{\lstset{
language=Python,
basicstyle={\footnotesize\ttfamily},
otherkeywords={self},             % Add keywords here
keywordstyle=\color{green},
emph={MyClass,__init__},          % Custom highlighting
emphstyle=\color{red},        % Custom highlighting style
stringstyle=\color{yellow},
showstringspaces=false            %
}}

% Python environment
\lstnewenvironment{python}[1][]
{
\pythonstyle
\lstset{#1}
}
{}

% Python for external files
\newcommand\pythonexternal[2][]{{
\pythonstyle
\lstinputlisting[#1]{#2}}}

% Python for inline
\newcommand\pythoninline[1]{{\pythonstyle\lstinline!#1!}}

%
}

% Configura a apresentação para ser executada em tela cheia.
\newcommand{\setfullscreen}{\hypersetup{pdfpagemode=FullScreen}}

% Hide beamer navigation simbols
\beamertemplatenavigationsymbolsempty

%
% Standard frames
%

% coverframe
\newcommand{\coverframe}{%
    \begin{frame} %
        \titlepage %
    \end{frame} %
}

% finalframe{email}
\newcommand{\finalframe}[1]{%
    \begin{frame}%
        \begin{flushright}%
            \huge \textbf{Thank you!}%
            \vfill%
            \large \textbf{#1}%
        \end{flushright}%
    \end{frame}%
}

%
% Useful styles.
%
\newcommand{\bigtitle}[1]{%
    \begin{center}%
        \Huge {#1}%
    \end{center}%
}


\presentationlanguage{brazil}
\loadtheme{apple_keynote_black}

\usecodingsamples{swift}

\title{Extensions}
\subtitle{The Swift Programming Language}
\author{Prof. Rafael Guterres Jeffman}
\institute{Faculdade Senac Porto Alegre\\
Especialização em Desenvolvimento para Dispositivos Móveis}
\date{Setembro de 2017}

\begin{document}

\coverframe

\begin{frame}
    \frametitle{Extensions}

    \textit{Extensions} permitem que se adicione novas funcionalidades
    a classes, \textit{structs}, \textit{enum}, ou protocolos.

    \vspace{0.4cm}

    Você pode utilizar \textit{extensions} mesmo quando não possui o
    código fonte de uma classe.

    \vspace{0.4cm}

    \textit{Extensions} \textbf{não} permitem que você modifique um
    comportamento já existente em uma classe.
\end{frame}

\begin{frame}[fragile]
    \frametitle{Sintaxe}
    \begin{swift}
    extension SomeType {
        // new functionality to add to SomeType goes here
    }

    extension SomeType: SomeProtocol, AnotherProtocol {
        // implementation of protocol requirements goes here
    }
    \end{swift}
\end{frame}

\begin{frame}[fragile]
    \frametitle{Atributos Calculados}
    \begin{swift}
    extension Double {
        var km: Double { return self * 1_000.0 }
        var m: Double { return self }
        var cm: Double { return self / 100.0 }
        var mm: Double { return self / 1_000.0 }
        var ft: Double { return self / 3.28084 }
    }
    let oneInch = 25.4.mm
    print("One inch is \(oneInch) meters")
    let threeFeet = 3.ft
    print("Three feet is \(threeFeet) meters")
    \end{swift}
\end{frame}

\begin{frame}[fragile]
    \frametitle{Métodos}
    \begin{swift}
        extension Int {
            func times(task: () -> Void) {
                for _ in 0..<self {
                    task()
                }
            }
        }

        3.times {
            print("Hello!")
        }
    \end{swift}
\end{frame}

\begin{frame}[fragile]
    \frametitle{Métodos Modificadores}
    \begin{swift}
    extension Int {
        mutating func square() {
            self = self * self
        }
    }
    var someInt = 3
    someInt.square()
    \end{swift}
\end{frame}

\begin{frame}[fragile]
    \frametitle{Índices}
    \begin{swift}
    extension Int {
        subscript(digitIndex: Int) -> Int {
            var decimalBase = 1
            for _ in 0..<digitIndex {
                decimalBase *= 10
            }
            return (self / decimalBase) % 10
        }
    }

    746381295[0] // returns 5
    746381295[1] // returns 9
    746381295[5] // returns 3
    \end{swift}
\end{frame}

\begin{frame}[fragile]
    \frametitle{Tipos de Dados Aninhados}
    \begin{swift}
    extension Int {
        enum Kind {
            case negative, zero, positive
        }
        var kind: Kind {
            switch self {
            case 0:
                return .zero
            case let x where x > 0:
                return .positive
            default:
                return .negative
            }
        }
    }
    \end{swift}
\end{frame}

\begin{frame}[fragile]
    \frametitle{Limites das Extensions}
    \begin{itemize}
        \setlength\itemsep{0.5cm}
        \item Ao adicionar novos comportamentos, eles estarão
        disponíveis a qualquer objeto daquela classe.
        \item \textit{Extensions} não podem adicionar armazenamento ao
        objeto, logo, só pode incluir atributos calculados.
        \item \textit{Extensions} são muito úteis, mas não devem ser
        abusadas!
    \end{itemize}
\end{frame}

\end{document}
