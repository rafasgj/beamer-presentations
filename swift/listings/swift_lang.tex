
\usepackage{listings}

\lstdefinelanguage{Swift}{
  %keyword1&2&6
  morekeywords = [1]{class, deinit, enum, extension, func, import,%
                     init, let, protocol, static, struct, subscript,%
                     typealias, var, public, private, open, internal,%
                     },
  %keyword3
  morekeywords = [2]{break, case, continue, default, do, else, %
                     fallthrough, for, if, in, return, switch, where,%
                     while, guard, defer, repeat%
                    },
  %keyword4
  morekeywords = [3]{as, dynamicType, is, new, super, self, Self, Type,%
                     try, catch, throw, true, false, nil, rethrows,
                     precondition%
                    },
  %keyword5
  morekeywords = [4]{associativity, didSet, get, infix, inout, left,%
                     mutating, none, nonmutating, operator, override,%
                     postfix, precedence, prefix, right, set, unowned,%
                     unowned(safe), unowned(unsafe), weak, willSet,%
                     required, strong, final, lazy, indirect, optional,%
                     print%
                    },
  morekeywords = [5]{iOS, macOS, \#available},%
  keywordstyle = [1]\color{green},
  keywordstyle = [2]\color{green},
  keywordstyle = [3]\color{green},
  keywordstyle = [4]\color{green},
  keywordstyle = [5]\color{cyan},
  sensitive = true,
  morecomment = [l]{//},
  morecomment = [s]{/*}{*/},
  morecomment = [s]{/**}{*/},
  commentstyle = \color{gray},
  morestring = [b]",
  morestring = [b]',
  stringstyle = \color{purple}
}

\newcommand\swiftstyle{\lstset{language=Swift,
    basicstyle={\footnotesize\ttfamily},
    stringstyle=\color{magenta!80!black},
    showstringspaces=false
}}

\lstnewenvironment{swift}[1][]
{
\swiftstyle
\lstset{#1}
}
{}

\newcommand\swiftexternal[2][]{{
\pythonstyle
\lstinputlisting[#1]{#2}}}

\newcommand\swiftinline[1]{{\pythonstyle\lstinline!#1!}}
