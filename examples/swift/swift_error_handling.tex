\documentclass[aspectratio=169,14pt,usenames,dvipsnames]{beamer}

\usepackage[utf8]{inputenc}
\usepackage{enumitem}
\usepackage{calc}

\newcommand{\loadtheme}[1]{%
    \input{../themes/#1}%
}
\newcommand{\presentationlanguage}[1]{%
    \usepackage[#1]{babel}%
}

\newcommand{\usecodingsamples}{%
    \usepackage{listings}%
    %
% You must use package 'listings' for this environment.
%

% Python style for highlighting
\newcommand\pythonstyle{\lstset{
language=Python,
basicstyle={\footnotesize\ttfamily},
otherkeywords={self},             % Add keywords here
keywordstyle=\color{green},
emph={MyClass,__init__},          % Custom highlighting
emphstyle=\color{red},        % Custom highlighting style
stringstyle=\color{yellow},
showstringspaces=false            %
}}

% Python environment
\lstnewenvironment{python}[1][]
{
\pythonstyle
\lstset{#1}
}
{}

% Python for external files
\newcommand\pythonexternal[2][]{{
\pythonstyle
\lstinputlisting[#1]{#2}}}

% Python for inline
\newcommand\pythoninline[1]{{\pythonstyle\lstinline!#1!}}

%
}

% Configura a apresentação para ser executada em tela cheia.
\newcommand{\setfullscreen}{\hypersetup{pdfpagemode=FullScreen}}

% Hide beamer navigation simbols
\beamertemplatenavigationsymbolsempty

%
% Standard frames
%

% coverframe
\newcommand{\coverframe}{%
    \begin{frame} %
        \titlepage %
    \end{frame} %
}

% finalframe{email}
\newcommand{\finalframe}[1]{%
    \begin{frame}%
        \begin{flushright}%
            \huge \textbf{Thank you!}%
            \vfill%
            \large \textbf{#1}%
        \end{flushright}%
    \end{frame}%
}

%
% Useful styles.
%
\newcommand{\bigtitle}[1]{%
    \begin{center}%
        \Huge {#1}%
    \end{center}%
}


\loadtheme{apple_keynote_black}
\presentationlanguage{brazil}

\usecodingsamples{swift}

\title{Tratamento de Erros na Linguagem Swift}
\author{Prof. Rafael Guterres Jeffman}
\institute{Faculdade Senac Porto Alegre\\
Especialização em Desenvolvimento para Dispositivos Móveis}
\date{Setembro de 2017}

\begin{document}

\coverframe

\begin{frame}
    \frametitle{Introdução}

    Tratamento de erros é o processo de responder e recuperar a execução
    do programa, frente a ocorrência de erros.

    \vspace{.7cm}

    Swift provê um excelente mecanismo de suporte para a geração,
    interceptação, propagação e manipulação de error recuperáveis em
    tempo de execução.
\end{frame}

\begin{frame}
    \frametitle{O protocolo Error}

    Em \textit{Swift} erros são representados por valores que atuam de
    acordo com o protocolo \texttt{Error}.

    \vspace{.7cm}

    Os {\color{green}\texttt{enum}} são excelentes formas de agrupar
    tipos de erros semelhantes.
\end{frame}

\begin{frame}[fragile]
    \frametitle{Definindo um conjunto de erros}

    \begin{swift}
    enum VendingMachineError: Error {
        case invalidSelection
        case insufficientFunds(coinsNeeded: Int)
        case outOfStock
    }
    \end{swift}

\end{frame}

\begin{frame}[fragile]
    \frametitle{Propagando erros}
    
    Para propagar erros, uma função deve declarar que ela propaga erros.
    
    \vspace{.5cm}
    Uma função que não declara que propaga erros deve tratá-los
    internamente.

    \begin{swift}
  func canThrowAnError() throws -> String

  func errorsWillBeHandled() -> String
    \end{swift}

\end{frame}

\begin{frame}[fragile]
    \frametitle{Função que pode gerar erro}
    \begin{swift}[basicstyle=\footnotesize]
class VendingMachine {
    func vend(itemNamed name: String) throws {
        guard let item = inventory[name] else
        { throw VendingMachineError.invalidSelection }
        guard item.count > 0 else
        { throw VendingMachineError.outOfStock }
        guard item.price <= coinsDeposited else {
            throw VendingMachineError.insufficientFunds(
                        coinsNeeded: item.price - coinsDeposited
                    )
        }
        ...
    }
}
    \end{swift}
\end{frame}

\begin{frame}[fragile]
    \frametitle{Invocando o método.}
    Ao invocar o método \texttt{buyFavoriteSnack}, se um erro ocorrer,
    ele será repassado adiante.

    \begin{swift}[basicstyle=\small]
func buyFavoriteSnack(person: String,
                      vendingMachine: VendingMachine) throws
{
    let snackName = favoriteSnacks[person] ?? "Candy Bar"
    try vendingMachine.vend(itemNamed: snackName)
}
    \end{swift}
\end{frame}

\begin{frame}[fragile]
    \frametitle{Tratando erros com \texttt{try-catch}}
    \begin{swift}[basicstyle=\footnotesize]
  do {
      try buyFavoriteSnack(person: "Alice",
                           vendingMachine: vendingMachine)
  } catch VendingMachineError.invalidSelection {
      print("Invalid Selection.")
  } catch VendingMachineError.outOfStock {
      print("Out of Stock.")
  } catch VendingMachineError.insufficientFunds(let coinsNeeded) {
      print("Please insert an additional \(coinsNeeded) coins.")
  }
    \end{swift}
\end{frame}

\begin{frame}[fragile]
    \frametitle{Erros e Optionals}

\begin{swift}
func someThrowingFunction() throws -> Int {
    // ...
}

let x = try? someThrowingFunction()
\end{swift}

\end{frame}

\begin{frame}[fragile]
    \frametitle{Ignorando erros}

    Quando você tem certeza que um método não vai falhar, você pode
    utilizar o {\color{green} \texttt try!} e ignorar qualquer erro. Se
    um erro ocorrer, a aplicação será encerrada.
    \vspace{.7cm}
    \begin{swift}[basicstyle=\small]
let photo = try! loadImage(atPath: "./John Appleseed.jpg")
    \end{swift}

\end{frame}

\begin{frame}[fragile]
    \frametitle{Liberando de recursos}
    \begin{swift}[basicstyle=\small]
  func processFile(filename: String) throws {
      if exists(filename) {
          let file = open(filename)
          defer {
              close(file)
          }
          while let line = try file.readline() {
              // Work with the file.
          }
          // close(file) is called here,
          // at the end of the scope.
      }
  }
    \end{swift}
\end{frame}

\begin{frame}[fragile]
    \frametitle{Defer}

    O {\color{green}\texttt{defer}} pode ser utilizado mesmo sem o
    tratamento de erros.

    \vspace{.75cm}

    Você pode definir diversos blocos {\color{green}\texttt{defer}},
    que serão executados em ordem inversa da definição (pilha).
\end{frame}

\end{document}
