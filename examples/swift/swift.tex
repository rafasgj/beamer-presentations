\documentclass[aspectratio=169,14pt,usenames,dvipsnames]{beamer}

\usepackage[utf8]{inputenc}
\usepackage{enumitem}
\usepackage{calc}

\newcommand{\loadtheme}[1]{%
    \input{../themes/#1}%
}
\newcommand{\presentationlanguage}[1]{%
    \usepackage[#1]{babel}%
}

\newcommand{\usecodingsamples}{%
    \usepackage{listings}%
    %
% You must use package 'listings' for this environment.
%

% Python style for highlighting
\newcommand\pythonstyle{\lstset{
language=Python,
basicstyle={\footnotesize\ttfamily},
otherkeywords={self},             % Add keywords here
keywordstyle=\color{green},
emph={MyClass,__init__},          % Custom highlighting
emphstyle=\color{red},        % Custom highlighting style
stringstyle=\color{yellow},
showstringspaces=false            %
}}

% Python environment
\lstnewenvironment{python}[1][]
{
\pythonstyle
\lstset{#1}
}
{}

% Python for external files
\newcommand\pythonexternal[2][]{{
\pythonstyle
\lstinputlisting[#1]{#2}}}

% Python for inline
\newcommand\pythoninline[1]{{\pythonstyle\lstinline!#1!}}

%
}

% Configura a apresentação para ser executada em tela cheia.
\newcommand{\setfullscreen}{\hypersetup{pdfpagemode=FullScreen}}

% Hide beamer navigation simbols
\beamertemplatenavigationsymbolsempty

%
% Standard frames
%

% coverframe
\newcommand{\coverframe}{%
    \begin{frame} %
        \titlepage %
    \end{frame} %
}

% finalframe{email}
\newcommand{\finalframe}[1]{%
    \begin{frame}%
        \begin{flushright}%
            \huge \textbf{Thank you!}%
            \vfill%
            \large \textbf{#1}%
        \end{flushright}%
    \end{frame}%
}

%
% Useful styles.
%
\newcommand{\bigtitle}[1]{%
    \begin{center}%
        \Huge {#1}%
    \end{center}%
}

\loadtheme{apple_keynote_black}

\presentationlanguage{brazil}

\usecodingsamples{swift}

\title{The Swift Programming Language}
\author{Prof. Rafael Guterres Jeffman}
\institute{Faculdade Senac Porto Alegre\\
Especialização em Desenvolvimento para Dispositivos Móveis}
\date{Setembro de 2017}

\begin{document}

\coverframe

\begin{frame}[fragile]
    \frametitle{Hello World}

    \begin{swift}
        print("Hello World! I'm Swift!")
    \end{swift}
\end{frame}

\bigtitle{The Basics}

\begin{frame}[fragile]
    \frametitle{Values}

    \begin{swift}
        // Constants
        let numberOfClasses = 5
        // Variables
        var totalSum = 0.0
        var i = 0, j = 0, k = 0
    \end{swift}
\end{frame}

\begin{frame}[fragile]
    \frametitle{Type Annotation}

    \begin{swift}
        // Constants
        let numberOfClasses: Int = 5
        // Variables
        var totalSum:: Double = 0.0
        var yourName: String
        var red, green, blue: Double
    \end{swift}
\end{frame}

\begin{frame}[fragile]
    \frametitle{Tuples}

    \begin{swift}
        let httpStatus = (404, "Page not found.")

        let status, message = httpStatus
    \end{swift}
\end{frame}

\begin{frame}[fragile]
    \frametitle{Optionals}
    \begin{itemize}
        \item \textbf{Optionals} quardam valores 'opcionais.'
        \item A variável com o tipo \textbf{Optional} ou possui um valor
        associado, ou \textbf{nil}.
        \item Para acessar o valor de um \textbf{Optional} deve-se
        "desempacotá-lo" utilzando o operadores \textbf{!} ou \textbf{?}.
        \item \textbf{Optional} é um tipo de dado genérico, e pode guardar
        apenas dados com o tipo de dado do dado associado.
    \end{itemize}
\end{frame}

\begin{frame}[fragile]
    \frametitle{Optionals}

    \begin{swift}
        let possibleNumber = "123"
        let convertedNumber = Int(possibleNumber)
        let optString: String? = nil
        let autoInt: Int! = Int("762")
    \end{swift}
    \vspace{1cm}
    \textit{convertNumber} é um \textbf{Optional$<$Int$>$} porque a conversão
    de uma \textbf{String} em um \textbf{Int} pode falhar.
\end{frame}

\begin{frame}[fragile]
    \frametitle{Optional Binding}

    \begin{swift}
        if let constantValue = Int("123") {
            // executed only when conversion works.
        }

        if let firstNumber = Int("4"),
            let secondNumber = Int("42"),
            firstNumber < secondNumber && secondNumber < 100
        {
            print("\(firstNumber) < \(secondNumber) < 100")
        }
    \end{swift}
\end{frame}

\begin{frame}[fragile]
    \frametitle{Operadores}

    \begin{itemize}
        \item Aritmeticos: + - * / \%
        \item Comparação: == != $> \; < \; >= \; <=$
        \item Ternário: $< condition > ? <truth\_value>\;:\;<false\_value> $
        \item Ternário Optional: $<Optional>\;??\;<default\_value> $
        \item Range: $0..{<}10$ $1...10$
        \item Lógino: $!\;\;||\;\;\&\&$
    \end{itemize}
\end{frame}

\begin{frame}[fragile]
    \frametitle{Strings}

    \begin{itemize}
        \item \textit{Strings} devem estar entre "$\,\!$"
        \item É possível criar \textit{strings} com múltiplas linhas
        utilizando "$\,\!$"$\,\!$"string "$\,\!$"$\,\!$"
        \item \textit{Strings} declaradas com \textbf{var} pdem ser
        alteradas, se declaradas com \textbf{let}, não podem
        \item \textit{Srings} são tipos de valores, ou seja, são copiados
        quando utilizados como argumentos.
    \end{itemize}
\end{frame}

\begin{frame}[fragile]
    \frametitle{Coleções}

    \begin{description}
        \item[Arrays] \hfill \\
        \begin{swift}
    var shoppingList: [String] = ["Eggs", "Milk"]
    shoppingList.append("Bread")
    let first = shoppingList[0]
    let lastItems = shoppingList[1...]
        \end{swift}
        \item[Dicionários] \hfill \\
        \begin{swift}
    var airports: [String: String] =
        ["YYZ": "Toronto Pearson", "DUB": "Dublin"]
    airports["LHR"] = "London"
    print(airports["YYZ"])  // Optional("Toronto Pearson")
        \end{swift}
    \end{description}
\end{frame}

\begin{frame}[fragile]
    \frametitle{Estruturas de Repetição: \textbf{for}}

    \begin{swift}
        for item in shoppingList {
            print(item)
        }
        for (key, value) in airports {
            print("\(key) : \(value)")
        }
        for i in 1...5 {
            print("The index is \(index)")
        }
        for second in 0..<60 {
            renderSecond(second)
        }
    \end{swift}
\end{frame}

\begin{frame}[fragile]
    \frametitle{Estruturas de Repetição: \textbf{while}}

    \begin{swift}
        while somethingIsTrue {
            doSomethingElse()
        }
    \end{swift}
\end{frame}

\begin{frame}[fragile]
    \frametitle{Estruturas de Repetição: \textbf{repeat}}

    \begin{swift}
        repeat
            thisCode()
        while isThis == true
        thisRunsWhenSomethingIsFalse()
    \end{swift}
\end{frame}

\begin{frame}[fragile]
    \frametitle{Estruturas de Decisão: \textbf{if}}

    \begin{swift}
        if thisValue % 2 == 0 {
            print ("\(thisValue) is even.")
        } else if thisValue % 3 == 0 {
            print ("\(thisValue) divisible by 3.")
        } else {
            print ("\(thisValue) is odd.")
        }
    \end{swift}
\end{frame}

\begin{frame}[fragile]
    \frametitle{Estruturas de Decisão: \textbf{switch}}

    \begin{swift}
        let someCharacter: Character = "z"
        switch someCharacter {
        case "a", "e", "i", "o", "u":
            print("\(someCharacter) is a vowel.")
        case "z":
            print("The last letter of the alphabet")
        default:
            print("Some other character")
        }
    \end{swift}
    \vspace{.5cm}
    Note que o \textbf{switch} deve tratar todos os valores possíveis
    para o tipo da variável, ou prover um \textbf{default}.
\end{frame}

\begin{frame}[fragile]
    \frametitle{Estruturas de Decisão: \textbf{switch}}

    \begin{swift}
        let approximateCount = 62
        switch approximateCount {
        case 0:
            naturalCount = "no"
        case 1..<5:
            naturalCount = "a few"
        case 5..<12:
            naturalCount = "several"
        case 12..<100:
            naturalCount = "dozens of"
        case 100..<1000:
            naturalCount = "hundreds of"
        default:
            naturalCount = "many"
        }
    \end{swift}
\end{frame}

\begin{frame}[fragile]
    \frametitle{Estruturas de Decisão: \textbf{switch}}

    \begin{swift}
        let somePoint = (1, 1)
        switch somePoint {
        case (0, 0):
            print("\(somePoint) is at the origin")
        case (_, 0):
            print("\(somePoint) is on the x-axis")
        case (0, _):
            print("\(somePoint) is on the y-axis")
        case (-2...2, -2...2):
            print("\(somePoint) is inside the box")
        default:
            print("\(somePoint) is outside of the box")
        }
    \end{swift}
\end{frame}

\begin{frame}[fragile]
    \frametitle{Estruturas de Decisão: \textbf{switch}}

    \begin{swift}
        let yetAnotherPoint = (1, -1)
        switch yetAnotherPoint {
        case let (x, y) where x == y:
            print("(\(x), \(y)) is on the line x == y")
        case let (x, y) where x == -y:
            print("(\(x), \(y)) is on the line x == -y")
        case let (x, y):
            print("(\(x), \(y)) is just some arbitrary point")
        }
    \end{swift}
\end{frame}

\begin{frame}[fragile]
    \frametitle{Transferência de Controle}

    \begin{description}[leftmargin=!,labelwidth=\widthof{\bfseries fallthrough}]
        \item[continue] Reinicia a iteração em um \textit{loop}.
        \item[break] Sai do \textit{loop} ou do \textbf{switch}.
        \item[fallthrough] Segue a execução no próximo caso do \textbf{switch}.
    \end{description}
\end{frame}

\begin{frame}[fragile]
    \frametitle{Funções}

    \begin{swift}
        func greet(person: String) -> String {
            let greeting = "Hello, " + person + "!"
            return greeting
        }

        greet(person: "John")
    \end{swift}
\end{frame}

\begin{frame}[fragile]
    \frametitle{Funções}

    \begin{swift}
    func greet(_ person: String, with greeting: String)->String
    {
        let greeting = "\(gretting), \(person)!"
        return greeting
    }

    greet("John", with: "Good morning")
    \end{swift}
\end{frame}

\begin{frame}[fragile]
    \frametitle{Funções}

    \begin{swift}
        func minMax(array: [Int]) -> (min: Int, max: Int) {
            var currentMin = array[0]
            var currentMax = array[0]
            for value in array[1..<array.count] {
                if value < currentMin {
                    currentMin = value
                } else if value > currentMax {
                    currentMax = value
                }
            }
            return (currentMin, currentMax)
        }
    \end{swift}
\end{frame}

\begin{frame}[fragile]
    \frametitle{um exemplo mais complexo.}

    \begin{swift}
        func chooseStepFunction(backward: Bool) -> (Int)->Int
        {
            func stepForward(input: Int) -> Int
            { return input + 1 }

            func stepBackward(input: Int) -> Int
            { return input - 1 }

            return backward ? stepBackward : stepForward
        }
    \end{swift}
\end{frame}

\begin{frame}[fragile]
    \frametitle{Closures}

    \begin{itemize}
        \item Swift foi desenvolvida utilizando diversas construções de
        linguagens funcionais.
        \item \textit{Closures} são uma forma leve de definir uma função
        no momento em que elas são necessárias.
    \end{itemize}
\end{frame}

\begin{frame}[fragile]
    \frametitle{Funções como parâmetros.}

    \begin{swift}
        func backward(_ s1: String, _ s2: String) -> Bool {
            return s1 > s2
        }

        var reversedNames = names.sorted(by: backward)
    \end{swift}
\end{frame}

\begin{frame}[fragile]
    \frametitle{Closures  como parâmetros.}

    \begin{swift}
    var reversedNames =
            names.sorted(by: {
               (_ s1: String, _ s2: String) -> Bool in
                    return s1 > s2
            })
    \end{swift}
\end{frame}

\begin{frame}[fragile]
    \frametitle{Inferência de Tipos}

    \begin{swift}
        var reversedNames = names.sorted(by: { s1, s2 in
                return s1 > s2
            })
    \end{swift}
\end{frame}

\begin{frame}[fragile]
    \frametitle{Retorno Implícito}

    \begin{swift}
        var reversedNames =
                names.sorted(by: { s1, s2 in s1 > s2 })
    \end{swift}
\end{frame}

\begin{frame}[fragile]
    \frametitle{Argumentos posicionais}

    \begin{swift}
        var reversedNames = names.sorted(by: { $0 > $1 })
    \end{swift}
\end{frame}

\begin{frame}[fragile]
    \frametitle{Operadores são funções}

    \begin{swift}
        var reversedNames = names.sorted(by: > )
    \end{swift}
\end{frame}

\begin{frame}[fragile]
    \frametitle{Enumerações}

    \begin{swift}
        enum Direction {
            case north
            case south
            case east
            case west
        }

        enum Direction {
            case north, south, east, west
        }
    \end{swift}
\end{frame}

\begin{frame}[fragile]
    \frametitle{Enumerações}

    Quando já se sabe o tipo de dado da enumeração, é possível
    utilizar apenas o valor desejado.
    \vspace{0.5cm}
    \begin{swift}
        var whereToGo = Direction.west
        // some code here
        whereToGo = .north
    \end{swift}
\end{frame}

\begin{frame}[fragile]
    \frametitle{Enumerações}

    \begin{swift}
        enum Operation {
            case constant (Double)
            case unary ((Double)->Double)
            case binary ((Double,Double)->Double)
            case equals
        }
    \end{swift}
\end{frame}

\begin{frame}[fragile]
    \frametitle{Enumerações}

    \begin{swift}
        let multiply = Operation.binary({ $0 * $1})

        switch operation {
        case constant (let value):
            print("Constant = \(value)")
        case unary (let ufn):
            print("Unary(5) = \(ufn(5))")
        case constant (let bfn):
            print("Binary(-2,7) = \(bfn(-2,7))")
        case equals:
            print("There's no associated value")
        }
    \end{swift}
\end{frame}

\begin{frame}[fragile]
    \frametitle{Classes and Structs}

    \begin{itemize}
        \item Definem propriedades para armazenar valores.
        \item Definem métodos para prover funcionalidades.
        \item Podem ser extendidas para adicionar novas funcionalidades.
        \item Podem implementar protocolos.
    \end{itemize}
\end{frame}

\begin{frame}[fragile]
    \frametitle{Classes}

    Classes, em Swift, são muito mais parecidas com \textbf{struct} do
    que em outras linguagens, mas possuem algumas características únicas:

    \begin{itemize}
        \item Herança pode ser aplicada a classes.
        \item Pode-se verificar e interpretar o tipo da classe em tempo
        de execução.
        \item Destrutores permitem a liberação de recursos alocados pela
        classe.
        \item Objetos de classes são referenciados e possuem contagem
        automática de referência.
    \end{itemize}
\end{frame}

\begin{frame}[fragile]
    \frametitle{Propriedades}

    \begin{swift}
        class AnotherClass {
            var aPositiveNumber = 0.0
            public var computedVariable: Double {
                get {
                    return aPositiveNumber
                }
                set {
                    if newValue > 0 { aPossitiveNumber = newValue}
                }
                willSet { print("Will set to \(newValue)" }
                didSet { print("Not \(oldValue) anymore.")}
            }
        }
    \end{swift}
\end{frame}

\begin{frame}[fragile]
    \frametitle{Métodos}

    \begin{swift}
        class Counter {
            var count = 0
            func increment() {
                count += 1
            }
            func increment(by amount: Int) {
                count += amount
            }
            func reset() {
                count = 0
            }
        }
    \end{swift}
\end{frame}

\begin{frame}[fragile]
    \frametitle{Optional Chaining}

    \begin{swift}
        class Person {
            var residence: Residence?
        }

        class Residence {
            var numberOfRooms = 1
        }

        if let roomCount = john.residence?.numberOfRooms {
            print("John's residence has \(roomCount) room(s).")
        } else {
            print("Unable to retrieve the number of rooms.")
        }
    \end{swift}
\end{frame}

\begin{frame}[fragile]
    \frametitle{Type Casting}

    \begin{swift}
    for item in library {
        if let movie = item as? Movie {
            print("Movie: \(movie.name), dir. \(movie.director)")
        } else if let song = item as? Song {
            print("Song: \(song.name), by \(song.artist)")
        }
    }
    \end{swift}
\end{frame}

\end{document}
