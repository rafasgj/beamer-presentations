\documentclass[aspectratio=169,14pt]{beamer}
\usepackage[utf8]{inputenc}
\usepackage[brazil]{babel}
\usepackage{graphics}
\usepackage{enumitem}

\usepackage{listings}

%
% You must use package 'listings' for this environment.
%

% Python style for highlighting
\newcommand\pythonstyle{\lstset{
language=Python,
basicstyle={\footnotesize\ttfamily},
otherkeywords={self},             % Add keywords here
keywordstyle=\color{green},
emph={MyClass,__init__},          % Custom highlighting
emphstyle=\color{red},        % Custom highlighting style
stringstyle=\color{yellow},
showstringspaces=false            %
}}

% Python environment
\lstnewenvironment{python}[1][]
{
\pythonstyle
\lstset{#1}
}
{}

% Python for external files
\newcommand\pythonexternal[2][]{{
\pythonstyle
\lstinputlisting[#1]{#2}}}

% Python for inline
\newcommand\pythoninline[1]{{\pythonstyle\lstinline!#1!}}


% set text colors
\setbeamercolor{frametitle}{fg=white}
\setbeamercolor{structure}{fg=white}
\setbeamercolor{normal text}{fg=white}
\setbeamercolor{alerted text}{fg=white}
\setbeamercolor{example text}{fg=white}

% set fonts
\setbeamerfont{title}{size=\huge}
\setbeamerfont{author}{size=\large}
\setbeamerfont{institute}{size=\small}
\setbeamerfont{date}{size=\footnotesize}
\setbeamerfont{frametitle}{size=\huge}

% set background gradient
\setbeamertemplate{background canvas}[vertical shading][top=black!85!gray,
                   bottom=black!60!gray,middle=black!85!gray,midpoint=0.35]

% set frame title posision.
\setbeamertemplate{frametitle}{\vspace{.8cm}$\,$\insertframetitle}

% set item markers
\setlist[itemize,1]{label={{\footnotesize$\blacktriangleright$}}}



% Configura a apresentação para ser executada em tela cheia.
%\hypersetup{pdfpagemode=FullScreen}
% Hide beamer navigation simbols
\beamertemplatenavigationsymbolsempty

%
% Modify from here!
%

\title{Linguagem de Programação\\Python}
\author{Prof. Me. Rafael Guterres Jeffman}
%\institute{Semana Acadêmica FURG}
\date{1$^o$ de Setembro de 2017}

\begin{document}

\begin{frame}
    \titlepage
\end{frame}

\begin{frame}
    \begin{center}
        \Huge Tipos de Dados
    \end{center}
\end{frame}

\begin{frame}
    \frametitle{Números}
    \begin{description}
        \setlength\itemsep{0.75em}
        \item[int] \hfill \\ Python pode utilizar números
        arbitrariamente grandes como 10**100 (Gogol).
        \item [float] \hfill \\ Utilizam grande precisão, logo 0.1 + 0.2
        não é igual a 0.3.
        \item [complex] \hfill \\ Python tem suporte nativo a números
        complexos.
    \end{description}
\end{frame}

\begin{frame}
    \frametitle{Strings}
    \begin{itemize}[label={}]
        \item \textit{Strings} são criadas utilizando \texttt{'$\ldots$'}
        ou \texttt{"$\ldots$"}.
        \item O tipo de dado criado é {\color{green}\texttt{str}}
        \item Strings de múltiplas linhas podem ser criadas com
        \texttt{'''$\ldots$'''} \\
            $\;\;\;$ \texttt{'''}Este é um texto de \\
            $\;\;\;$ múltiplas linhas.\texttt{'''}
        \item Quando não se deseja a criação de caracteres especiais,
        utiliza-se a \textit{string raw} \\
            $\;\;\;$r'C:\textbackslash trash'
    \end{itemize}
\end{frame}

\begin{frame}
    \frametitle{Listas}
    \begin{description}
        \item[l = {$[1,2,3,4]$}] \hfill \\ Cria uma lista com os elementos $[1,2,3,4]$
        \item[l{$[0]$}] \hfill \\ Acessa primeiro elemento da lista
        \item[l{$[-1]$}] \hfill \\ Acessa último elemento da lista.
        \item[l = {$[0]*10$}] \hfill \\ Cria uma lista com 10 elementos $0$.
    \end{description}
\end{frame}

\begin{frame}
    \frametitle{\textit{Slicing}}
    \begin{description}
        \item[l{$[0:2]$}] \hfill \\ Cria uma lista com os elementos $[1,2]$
        \item[l{$[2:-1]$}] \hfill \\ Cria uma lista com os elementos $[3,4]$
        \item[l{$[:]$}] \hfill \\ Duplica uma lista.
    \end{description}
\end{frame}

\begin{frame}
    \begin{center}
        \Huge Estruturas de Controle
    \end{center}
\end{frame}

\begin{frame}[fragile]
    \frametitle{if - elif - else}
    \begin{python}
        if condicao:
            verdade
        else:
            falso
    \end{python}
\end{frame}

\begin{frame}[fragile]
    \frametitle{while}
    \begin{python}
        while condicao:
            codigo
        else:
            codigo executado se condicao falsa.
    \end{python}
\end{frame}

\begin{frame}[fragile]
    \frametitle{for - in}
    \begin{python}
        for variavel in range:
            codigo
        else:
            codigo executado apos todos elementos
    \end{python}
\end{frame}

\begin{frame}
    \frametitle{continue, break}
    \texttt{\color{green}continue} reinicia o laço com o próximo estado.
    \texttt{\color{green}break} sai do laço, ignorando a clausula
    \texttt{\color{green}else}
\end{frame}

\begin{frame}
    \begin{center}
        \Huge Funções
    \end{center}
\end{frame}

\begin{frame}[fragile]
    \frametitle{Definições}
    \begin{python}
        def fib(n):
             """Compute the n-th Fibonacci number."""
             a, b = 0, 1
             for _ in range(n):
                 a, b = b, a+b
             return a
    \end{python}
\end{frame}

\begin{frame}
    \frametitle{Parâmetros}
    \begin{itemize}
        \item A passagem de parâmetros é por valor. O valor de
        objetos passados por parâmetro é a referência ao objeto.
        \item Não é necessário observar a ordem dos parâmetros,
        desde que sejam utilizados seus nomes.
        \item Não há sobrecarga de funções em Python, para simular
        a sobrecarga, utilize parâmetros com valores \textit{default}.
        \item Existe uma certa beleza na utilização de número de
        parâmetros variáveis em Python, mas está fora do escopo desta
        apresentação.
    \end{itemize}
\end{frame}

\begin{frame}[fragile]
    \frametitle{Escopo}
    \begin{python}
        def scope_test():
            def do_local():
                spam = "local spam"

            def do_nonlocal():
                nonlocal spam
                spam = "nonlocal spam"

            def do_global():
                global spam
                spam = "global spam"
    \end{python}
\end{frame}

\begin{frame}
    \begin{center}
        \Huge Programação Funcional
    \end{center}
\end{frame}

\begin{frame}
    \footnotesize
    \begin{description}
        \item[First-class Functions] \hfill \\ As funcões de Python podem ser
        tratadas como valores e passadas como parâmetros.
        \item[lambda] \hfill \\  É possível criar funções anônimas e utilizá-las
        onde se espera uma função qualquer.
        \item[map]  \hfill \\ Aplica uma função a cada um dos elementos de uma lista.
        \item[filter]  \hfill \\ Filtra uma lista baseado no resultado de um predicado.
        \item[functools.reduce]  \hfill \\ Aplica uma função a uma lista, passando como
        parâmetros o próximo elemento da lista e o resultado parcial.
        \item[sum]  \hfill \\ Soma todos os elementos de uma lista.
    \end{description}
\end{frame}

\begin{frame}[fragile]
    \frametitle{Exemplo de Programação Funcional}
    \begin{python}
        map(lambda x: x**2,
            filter(lambda y: y % 3 == 0, range(25))
        )
    \end{python}
\end{frame}


\begin{frame}
    \begin{center}
        \Huge Programação Orientada a Objetos
    \end{center}
\end{frame}
\begin{frame}
    \frametitle{Classes}
    \begin{itemize}[label={}]
        \setlength\itemsep{1.5em}
        \item Python é uma linguagem orientada a objetos.
        \item É utilizado o conceito de classes.
        \item A orientação a objetos em Python leva ao extremo o
        lema "explícito é melhor que implícito".
    \end{itemize}
\end{frame}

\begin{frame}[fragile]
    \frametitle{Exemplo de Classe}
    \begin{python}
        class Complex:
            def __init__(self, realpart, imagpart):
                self.r = realpart
                self.i = imagpart
            def has_imaginary(self):
                return self.imagpart != 0
    \end{python}
\end{frame}

\begin{frame}[fragile]
    \frametitle{Herança}
    \begin{itemize}
        \item Python implementa o conceito de herança múltipla.
        \item A primeira classe em uma lista de herança é "mais
        importante" do que as outras.
        \item Todo método é um método virtual.
    \end{itemize}
    \begin{python}
        class Veiculo(object):
            pass
        class Terrestre(Veiculo):
            pass
        class Aquatico(Veiculo):
            pass
        class Hovercraft(Terrestre, Aquatico):
            pass
    \end{python}
\end{frame}

\begin{frame}
    \begin{center}
        \Huge Estruturas de Dados
    \end{center}
\end{frame}

\begin{frame}[fragile]
    \frametitle{Pilhas}
    Listas funcionam muito bem como pilhas (\textit{stacks}):
    \begin{python}
        stack = []
        stack.append(5)
        stack.append(2)
        stack.pop()
        stack.pop()
    \end{python}
\end{frame}

\begin{frame}[fragile]
    \frametitle{Filas}
    Listas são ineficientes para a utilização como filas, devido à
    inserção no início da lista ser lenta. Utilize
    \texttt{\color{green}collections.deque}
    \begin{python}
        from collections import deque
        queue = deque([])
        deque.append("Eric")
        deque.append("Graham")
        deque.popleft()
        deque.popleft()
    \end{python}
\end{frame}

\begin{frame}[fragile]
    \frametitle{List Comprehension}
    Forma concisa de criar listas baseadas em condições. Normalmente
    mais eficiente que laços implementados \textit{à mão}.
    \begin{python}
        squares = [ x**2 for x in range(10) ]
        different_pairs = [ (x,y) for x in [1,2,3]
                                  for y in [1,2,3] if x != y ]

        l = [ [1,2,3], [4,5,6], [7,8,9]]

        flat = [ x for sub in l for x in sub ]
        matrix_3x3 = [ [0]*3 for _ in range(3) ]
    \end{python}
\end{frame}

\begin{frame}[fragile]
    \frametitle{Nested List Comprehension}
    \begin{python}
        matrix = [
            [1, 2, 3, 4],
            [5, 6, 7, 8],
            [9, 10, 11, 12],
        ]

        [[row[i] for row in matrix]
            for i in range(len(matrix[0]))]
    \end{python}
\end{frame}

\begin{frame}[fragile]
    \frametitle{Conjuntos}
    \begin{python}
        basket = {'apple', 'orange', 'apple',
                  'pear', 'orange', 'banana'}

        'orange' in basket

        a = set('abracadabra')
        b = set('alacazam')
        a - b
        a | b
        a & b
        a ^ b
    \end{python}
\end{frame}

\begin{frame}[fragile]
    \frametitle{Dicionários}
    \begin{python}
        tel = {'jack': 4098, 'sape': 4139}
        tel['guido'] = 4127
        del tel['sape']
        'jack' not in tel

        squares = { x: x**2 for x in range(5) }

        tel = dict(jack=4098, sape=4139, guido=4127)
    \end{python}
\end{frame}

\begin{frame}
    \begin{center}
        \Huge Técnicas para Repetições
    \end{center}
\end{frame}

\begin{frame}[fragile]
    \frametitle{Iteração sobre Dicionários}
    \begin{python}
        for k, v in knights.items():
             print(k, v)

        for k in knights.keys():
             print(k, knights[v])
    \end{python}
\end{frame}

\begin{frame}[fragile]
    \frametitle{Sequências com Índices}
    \begin{python}
        for i, v in enumerate(["tic","tac","toe"])]:
             print(i, v)
    \end{python}
\end{frame}

\begin{frame}[fragile]
    \frametitle{Sequências Simultâneas}
    \begin{python}
         questions = ['name', 'quest', 'favorite color']
         answers = ['lancelot', 'the holy grail', 'blue']

         for q, a in zip(questions, answers):
              print('What is your {0}?  It is {1}.'.format(q, a))
    \end{python}
\end{frame}

\begin{frame}[fragile]
    \frametitle{Outras formas de iteração.}
    \begin{python}
        for i in reversed(range(1, 10, 2)):
            print(i)

        for i in sorted(['orange', 'apple', 'pear', 'banana']):
            print(i)
    \end{python}
\end{frame}

\begin{frame}
    \begin{center}
        \Huge Estruturas de Dados Avançadas
    \end{center}
\end{frame}

\begin{frame}[fragile]
    \frametitle{Classe Árvore Binária de Pesquisa}
    \begin{python}
class BSTNode (object):
    def __init__(self, value):
        self.left = None
        self.right = None
        self.value = value
    def insert(self, value):
        if value < self.value:
            if self.left == None: self.left = BSTNode(value)
            else: self.left = self.left.insert(value)
        elif value > self.value:
            if self.right == None: self.right = BSTNode(value)
            else: self.right = self.right.insert(value)
        return self
    \end{python}
\end{frame}
\begin{frame}
    \frametitle{Operador Ternário em Python}
    \begin{center}
        \normalsize\ttfamily
        value\_true {\color{green}if} boolean condition {\color{green}else} value\_false
    \end{center}
\end{frame}
\begin{frame}[fragile]
    \frametitle{Revisão da ABP}
    \begin{python}
    def insert(self, value):
        do_insert = lambda n,v: BSTNode(v) if n == None \
                                else n.insert(v)
        if value < self.value:
            self.left = do_insert(self.left, value)
        elif value > self.value:
            self.right = do_insert(self.right, value)
        return self
    \end{python}
\end{frame}
\begin{frame}[fragile]
    \frametitle{ABP Funcional Insert}
    \begin{python}
    from functools import reduce

    bst_insert = lambda n, v: \
            node(v) if n == None else \
              node(n.value,bst_insert(n.left,v),n.right) \
                if v <= n.value else \
                  node(n.value, n.left, bst_insert(n.right,v))

    bst = reduce(bst_insert, [ 3, 5, 2, 1, 4, 6 ], None)
    \end{python}
\end{frame}

\begin{frame}
    \begin{center}
        \Huge Implementação de Algoritmos
    \end{center}
\end{frame}

\begin{frame}[fragile]
    \frametitle{Quicksort}
    \begin{python}
     def quicksort(n):
         def __quicksort(n,s,e):
             if s <  e:
                 n[s:e], b, a = partition(n[s:e], pivot(n[s:e]))
                 __quicksort(n,s,s+b)
                 __quicksort(n,s+a,e)
             return n

         __quicksort(n,0,len(n))
    \end{python}
\end{frame}

\begin{frame}[fragile]
    \frametitle{Dijkstra's Tree-way Partition}
    \begin{python}
    def partition(a, mid):
        i,j,n = 0,0,len(a) - 1
        while j <= n:
            if a[j] < mid:
                a[i], a[j] = a[j], a[i]
                i, j = i+1, j+1
            elif a[j] > mid:
                a[j],a[n] = a[n],a[j]
                n -= 1
            else: j += 1
        return a,i,j
    \end{python}
\end{frame}

\begin{frame}[fragile]
    \frametitle{Median-of-Tree Pivot Selection}
    \begin{python}
    def pivot(n):
        mid = len(n) // 2
        mx = max(n[0],n[mid],n[-1])
        mn = min(n[0],n[mid],n[-1])
        if mn <= n[0] <= mx: return n[0]
        if mn <= n[-1] <= mx: return n[-1]
        return n[mid]
    \end{python}
\end{frame}

\begin{frame}
    \begin{center}
        \Huge Uma Aplicação de Exemplo
    \end{center}
\end{frame}

\begin{frame}
    \huge{Where do we go from here?}
\end{frame}

\begin{frame}
    \frametitle{Quais os próximos passos?}
    \begin{itemize}[label={}]
        \item Módulos.
        \item Entrada e Saída por arquivos.
        \item Tratamento de erros e exceções.
        \item Python Standard Library.
        \item Pip.
        \item Django, Flask and the Web.
    \end{itemize}
\end{frame}

\begin{frame}
    \frametitle{E onde estão as informações?}
    \centering \Large https://docs.python.org/3/tutorial/index.html
\end{frame}

\begin{frame}
    \begin{flushright}
    \huge \textbf{Thank you!} \\
    \vfill
    \small \textbf{rafasgj@gmail.com}
    \vfill
    \small \textbf{https://github.com/rafasgj/python-furg-2017.git}
    \end{flushright}
\end{frame}

\end{document}
